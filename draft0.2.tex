%%
%% This is file `mcmthesis-demo.tex',
%% generated with the docstrip utility.
%%
%% The original source files were:
%%
%% mcmthesis.dtx  (with options: `demo')
%% !Mode:: "TeX:UTF-8"
%% -----------------------------------
%%
%% This is a generated file.
%%
%% Copyright (C)
%%     2010 -- 2015 by latexstudio
%%     2014 -- 2016 by Liam Huang
%%
%% This work may be distributed and/or modified under the
%% conditions of the LaTeX Project Public License, either version 1.3
%% of this license or (at your option) any later version.
%% The latest version of this license is in
%%   http://www.latex-project.org/lppl.txt
%% and version 1.3 or later is part of all distributions of LaTeX
%% version 2005/12/01 or later.
%%
%% This work has the LPPL maintenance status `maintained'.
%%
%% The Current Maintainer of this work is Liam Huang.
%%
\documentclass{mcmthesis}
\mcmsetup{CTeX = true,   % 使用 CTeX 套装时,设置为 true
        tcn = 1911507, problem = B,
        sheet = true, titleinsheet = true, keywordsinsheet = true,
        titlepage = false, abstract = true}
\usepackage{palatino}
\usepackage{lipsum}
\title{The \LaTeX{} Template for MCM Version \MCMversion}
\author{\small \href{http://www.latexstudio.net/}
  {\includegraphics[width=7cm]{mcmthesis-logo}}}
\date{\today}
\usepackage{geometry}
\geometry{a4paper,left=2.6cm,right=2.6cm,top=3cm,bottom=3cm}
\begin{document}
\begin{abstract}
\lipsum[1]
\begin{keywords}
keyword1; keyword2
\end{keywords}
\end{abstract}
\maketitle
\tableofcontents
\newpage

\section{Introduction}
\subsection{Problem Background}
Natural disasters often bring huge casualties and economic losses on human society. In 2017, the worst hurricane in nearly a century landed in Puerto Rico, and left almost the entire island without power, and many without running water or cell phone service. Many highways and roads were blocked because of widespread flooding, which brought great difficulties to rescue route planning. Demand for medical supplies was also soaring for some time. \\

\noindent HELP, Inc., one of non-governmental organizations, is planning to design a transportable disaster response system called DroneGo to conduct medical supply delivery and video reconnaissance.

\subsection{Restatement of Problem}

Our task is to design a DroneGo disaster response system to support potential future disaster scenario similar as the Puerto Rico hurricane. Based on the 2017 situation in Puerto Rico, we will: 

\begin{itemize}
	\item Provide the packing configuration for each of no more than three ISO cargo containers to transport the system.
	\item Find the best locations on Puerto Rico to position these cargo containers to realize both medical supply delivery and video reconnaissance.
	\item Design the drone payload packing configurations, delivery routes and schedule to conduct medical supply delivery, and a drone flight plan to realize video reconnaissance of road networks.
\end{itemize}



\section{General Assumptions}
\textbf{Assumption I}: DroneGo disaster response system consists of up to three ISO cargo containers, a DroneGo fleet and some emergency medical packages. 

\noindent The three cargo containers are identical with standard size, designated as Container A, Container B and Container C respectively. The DroneGo fleet is a combination of drones selected from eight types of potential candidates, namely Drone A to Drone H. Only three types of emergency medical packages are available, referred to as MED 1, MED 2, and MED 3.

\noindent\textbf{Assumption II}: DroneGo disaster response system is designed for possible Puerto Rico hurricane disaster. 

\noindent When disasters occurs, a DroneGo fleet and some emergency medical packages will be packed in up to three cargo containers first of all. 

\noindent Next, each cargo container will be transported to one of the 32 populated places, which are represented by yellow square in Attachment 1, to ensure that DroneGo disaster response system can be timely discovered and well operated. 

\noindent Then both drones and medical packages will be taken out of the containers, and afterwards the latter will be packed into the drone cargo bay that is in fixed connection with the drone. 

\noindent Finally, all drones will depart from the up to three container locations and fly along the main roads on schedule. As is shown in Attachment 1, these main roads connect 32 populated places and make a road network. It is worth pointing out that there are two possible situations for these drones. If the drone carries a cargo bay with medical packages in it, the drone must fly via five locations in need of medical assistance, referred to as five destinations: Jajardo, San Pablo, San Juan, Bayamon and Arecibo, for the purpose of offloading its cargo. However, if the drone carries no cargo, any route without deviation of the road network is allowed.

\section{Symbols and definitions}\label{section:Sym}
\renewcommand\arraystretch{1.5}
\begin{tabular}{ll}
	\hline
	Symbol&  Definition\\
	\hline
	$X$& The number of the containers $X$.\\
	
	$Y$& The number of the containers $Y$.\\
	
	$Z$& The number of the containers $Z$.\\
	
	$X_\alpha$& The number of the drones $\alpha$ packed in the container $X$.\\
	
	$Y_\alpha$& The number of the drones $\alpha$ packed in the container $Y$.\\
	
	$Z_\alpha$& The number of the drones $\alpha$ packed in the container $Z$.\\
	
	$X_{\alpha ij}$& The number of the MED $i$ transported by drone $\alpha$ from\\	 
	& container $X$'s location to the demand location $j$, where $\alpha=A,\cdots,H$,\\ 
	& $i=1,2,3$, $j=1,2,\cdots,5$.\\
	
	$Y_{\alpha ij}$& The number of the MED $i$ transported by drone $\alpha$ from \\
	&container $Y$'s location to the demand location $j$, where $\alpha=A,\cdots,H$,\\
	& $i=1,2,3$, $j=1,2,\cdots,5$.\\
	
	$Z_{\alpha ij}$& The number of the MED $i$ transported by drone $\alpha$ from \\
	&container $Z$'s location to the demand location $j$, where $\alpha=A,\cdots,H$,\\ 
	&$i=1,2,3$, $j=1,2,\cdots,5$.\\
	
	$c_i$& The cost of MED $i$, where $i=1,2,3$.\\
	
	$C_\alpha$& The cost of Drone $\alpha$, where $\alpha=A,B,\cdots,H$.\\
	
	 $W$& The cost of the container.\\
	
	 $V_\alpha$& The volume of the drone $\alpha$, where $\alpha=A,B,\cdots,H$.\\
	
	 $v_i$& The volume of the medical package $i$, where $i=1,2,3$.\\
	
	 $P_\alpha$& The max payload capability of the drone $\alpha$, where $\alpha=A,B,\cdots,H$.\\
	
	 $d$& The supporting days for the medical package demand of all 5 locations.\\
	
	 $M$& A sufficiently large number for big-$M$ method in integer programming.\\ 
	\hline
\end{tabular}


\begin{tabular}{ll}
	\hline
	Symbol&  Definition\\
	\hline	 
	 $R$& A parameter indicating the significance of the video reconnaissance of\\
	 & road networks compared to medical supply delivery. \\
	 
	 $l_\alpha$& The max flight distance of the drone $\alpha$.\\
	 
	 $T$& The number of the nodes in the network.\\
	 
	 $U_{Xm}$& $U_{Xm}=1$ if container $X$ is located at node $m$, where $m=1,2,\cdots,T$.\\ & Otherwise $U_{Xm}=0$.\\
	 
	 $U_{Ym}$& $U_{Ym}=1$ if container $Y$ is located at node $m$, where $m=1,2,\cdots,T$.\\ &Otherwise $U_{Ym}=0$.\\
	 
	 $U_{Xm}$& $U_{Zm}=1$ if container $Z$ is located at node $m$, where $m=1,2,\cdots,T$.\\ &Otherwise $U_{Zm}=0$.\\
	 
	$f(m,j)$& The length of the shortest path from node $m$ to node $j$.\\
	\hline
\end{tabular}

\section{Model 1}
The DroneGo disaster response system is sent to Puerto Rico for two major missons: medical supply delivery and video reconnaissance of road networks, none of which is dispensable. Nevertheless, to start with, focusing on the former solely helps attain a clear insight into this sophisticated problem. That leads to our integer programming model.

\noindent Now assume that the response system are only for medical supply delivery and that the location of each container is so close to its destination that each drone can reach it before the power supply runs out. According to the requirement, the response system must meet the daily medical package demand of the five selected locations in Puerto Rico. Therefore, a natural objective is to maximize the so-called supporting days $d$. Suppose that the amount of the medical packages provided by the Drone fleet can serve location $j$ $(j=1,2,\cdots,5)$ for $d_j$ days. Then the supporting days is defined as
\begin{equation}
	d=\min\{d_1,d_2,d_3,d_4,d_5\}.
\end{equation}
For the sake of the optimization of $d$, a lot of independent variables need to be specified appropriately. Firstly, we need to specify the number of the containers transported to the disaster area. We introduce binary variables $X$, $Y$, $Z$ to indicate whether the corresponding cargo container is utilized respectively. Secondly, we have to decide the packing configuration for each container in use. Given the definition of $X_\alpha$ and $X_{\alpha ij}$ in section \ref{section:Sym}, the number of each item to be packed into container $X$ (8 types of drones: Drone A to Drone H; 3 types of medical packages: MED1, MED2, MED3) can be expressed as
 
\begin{equation*}
	X_A,\cdots,X_H,\sum_{\alpha=A}^H\sum_{j=1}^{5}X_{\alpha 1j},\sum_{\alpha=A}^H\sum_{j=1}^{5}X_{\alpha 2j},\sum_{\alpha=A}^H\sum_{j=1}^{5}X_{\alpha 3j}
\end{equation*} 
respectively. Likewise, we can choose values for $Y_\alpha,Y_{\alpha ij}$ and $Z_{\alpha},Z_{\alpha ij}$ to give the packing configuration for the container $Y$ and the container $Z$ respectively. Thirdly, we have to decide the drone payload packing configurations, or alternatively how the medical packages are packed into the drone cargo bay. The number of each type of medical package to be packed into the drone $\alpha$ can be expressed as
\begin{equation*}
\sum_{j=1}^{5}X_{\alpha 1j},\sum_{j=1}^{5}X_{\alpha 2j},\sum_{j=1}^{5}X_{\alpha 3j}
\end{equation*} Finally, the destination of each drone should be given. That is exactly what the variables $X_{\alpha ij},Y_{\alpha ij},Z_{\alpha ij}$ indicate. In a word, we need to adjust the 9 variables
\[
X,Y,Z,X_{\alpha},Y_{\alpha},Z_{\alpha},X_{\alpha ij},Y_{\alpha ij},Z_{\alpha ij}
\]
for the maximization of $d$.

\noindent By imposing proper constraints we obtain the following integer programming problem

\[
\begin{aligned}
\max\quad d
\end{aligned}
\]
\[
\begin{aligned}
\text{s.t.}
\left\{
\begin{array}{lr}
\sum\limits_{i=1}^{3}\sum\limits_{j=1}^{5}v_{i}X_{\alpha ij}\le 8\times 10\times 14\ \text{ if }\alpha=A,B,D &(1)\\
\sum\limits_{i=1}^{3}\sum\limits_{j=1}^{5}v_{i}X_{\alpha ij}\le 24\times 20\times 20\ \text{ if }\alpha=C,E,F,G &(2)\\
\sum\limits_{i=1}^{3}\sum\limits_{j=1}^{5}v_{i}Y_{\alpha ij}\le 8\times 10\times 14\ \text{ if }\alpha=A,B,D &(3)\\
\sum\limits_{i=1}^{3}\sum\limits_{j=1}^{5}v_{i}Y_{\alpha ij}\le 24\times 20\times 20\ \text{ if }\alpha=C,E,F,G &(4)\\
\sum\limits_{i=1}^{3}\sum\limits_{j=1}^{5}v_{i}Z_{\alpha ij}\le 8\times 10\times 14\ \text{ if }\alpha=A,B,D &(5)\\
\sum\limits_{i=1}^{3}\sum\limits_{j=1}^{5}v_{i}Z_{\alpha ij}\le 24\times 20\times 20\ \text{ if }\alpha=C,E,F,G &(6)\\
\sum\limits_{\alpha=A}^{H}V_{\alpha}X_{\alpha}+\sum\limits_{\alpha=A}^{H}\sum\limits_{i=1}^{3}\sum\limits_{j=1}^{5}v_{i}X_{\alpha ij}\le 231\times92\times 94&(7)\\
\sum\limits_{\alpha=A}^{H}V_{\alpha}Y_{\alpha}+\sum\limits_{\alpha=A}^{H}\sum\limits_{i=1}^{3}\sum\limits_{j=1}^{5}v_{i}Y_{\alpha ij}\le 231\times92\times 94&(8)\\
\sum\limits_{\alpha=A}^{H}V_{\alpha}Z_{\alpha}+\sum\limits_{\alpha=A}^{H}\sum\limits_{i=1}^{3}\sum\limits_{j=1}^{5}v_{i}Z_{\alpha ij}\le 231\times92\times 94&(9)\\
\sum\limits_{i=1}^{3}\sum\limits_{j=1}^{5}X_{\alpha ij}\le M X_{\alpha}\ \text{ for }\alpha=A,B,\cdots, G&(10)\\
\sum\limits_{i=1}^{3}\sum\limits_{j=1}^{5}Y_{\alpha ij}\le M X_{\alpha}\ \text{ for }\alpha=A,B,\cdots, G&(11)\\
\sum\limits_{i=1}^{3}\sum\limits_{j=1}^{5}Z_{\alpha ij}\le M X_{\alpha}\ \text{ for }\alpha=A,B,\cdots, G&(12)\\
\sum\limits_{\alpha=A}^{H}X_\alpha\le MX&(13)\\
\sum\limits_{\alpha=A}^{H}Y_\alpha\le MY&(14)\\
\sum\limits_{\alpha=A}^{H}Z_\alpha\le MZ&(15)\\
X\le 1&(16)\\
Y\le 1&(17)\\
Z\le 1&(18)\\
\sum\limits_{j=1}^{5}\left(2X_{\alpha 1 j}+2X_{\alpha 2 j}+3X_{\alpha 3 j}\right)\le P_\alpha\text{ for }\alpha=A,B,\cdots, G&(19)\\
\sum\limits_{j=1}^{5}\left(2Y_{\alpha 1 j}+2Y_{\alpha 2 j}+3Y_{\alpha 3 j}\right)\le P_\alpha\text{ for }\alpha=A,B,\cdots, G&(20)\\
\sum\limits_{j=1}^{5}\left(2Z_{\alpha 1 j}+2Z_{\alpha 2 j}+3Z_{\alpha 3 j}\right)\le P_\alpha\text{ for }\alpha=A,B,\cdots, G&(21)\\
\end{array}
\right.
\end{aligned}
\]
\[
\begin{aligned}
\text{s.t.}&
\left\{
\begin{array}{lr}
\sum\limits_{\alpha=A}^{H}(X_{\alpha 11}+Y_{\alpha 11}+Z_{\alpha 11})\ge d&(22)\\
\sum\limits_{\alpha=A}^{H}(X_{\alpha 31}+Y_{\alpha 31}+Z_{\alpha 31})\ge d&(23)\\
\sum\limits_{\alpha=A}^{H}(X_{\alpha 12}+Y_{\alpha 12}+Z_{\alpha 12})\ge 2d&(24)\\
\sum\limits_{\alpha=A}^{H}(X_{\alpha 32}+Y_{\alpha 32}+Z_{\alpha 32})\ge d&(25)\\
\sum\limits_{\alpha=A}^{H}(X_{\alpha 13}+Y_{\alpha 13}+Z_{\alpha 13})\ge d&(26)\\
\sum\limits_{\alpha=A}^{H}(X_{\alpha 23}+Y_{\alpha 23}+Z_{\alpha 23})\ge d&(27)\\
\sum\limits_{\alpha=A}^{H}(X_{\alpha 14}+Y_{\alpha 14}+Z_{\alpha 14})\ge 2d&(28)\\
\sum\limits_{\alpha=A}^{H}(X_{\alpha 24}+Y_{\alpha 24}+Z_{\alpha 24})\ge d&(29)\\
\sum\limits_{\alpha=A}^{H}(X_{\alpha 34}+Y_{\alpha 34}+Z_{\alpha 34})\ge 2d&(30)\\
\sum\limits_{\alpha=A}^{H}(X_{\alpha 15}+Y_{\alpha 15}+Z_{\alpha 15})\ge d&(31)\\
X,Y,Z,X_\alpha,Y_\alpha,Z_\alpha,X_{\alpha ij},Y_{\alpha ij},Z_{\alpha ij} \text{ are nonnegative integers.}&(32)

\end{array}
\right.
\end{aligned}
\]

\noindent If we take the cost of containers, drones and medical packages into consideration and fix the value of $d$, minimizing the total cost also makes sense. To address the minimization problem, we only need to transform the objective function into the following form while all original constraints (1)-(32) still hold:
\[
\begin{aligned}
&\min\quad\sum_{\alpha=A}^{H}\sum_{i=1}^{3}\sum_{j=1}^{5}c_{i}\left(X_{\alpha ij}+Y_{\alpha ij}+Z_{\alpha ij}\right)
+\sum_{\alpha=A}^{H}C_{{\alpha}}\left(X_{\alpha}+Y_{\alpha }+Z_{\alpha }\right)+W(X+Y+Z).
\end{aligned}
\]


\section{Model 2}

Now it is time to extend our analysis to the complete case that both medical supply delivery and video reconnaissance of road networks need to be carried out. Since the variable $d$ named supporting days has been shown to be a reasonable measure of the effect of medical supply delivery, developing some appropriate measures of the effect of video reconnaissance is our priority.

\noindent Assume that drones cannot be recharged, which implies each drone will fly along the road network until its batter is discharged. When the number of drones is rather large, it is totally reasonable to assume that under no circumstances will any two drones fly abreast along a same road. Thus the sum of flight distances of all the drones can well measure the effect of video reconnaissance.
\[
\begin{aligned}
\max\quad d+R\sum_{\substack{\alpha=A\\ \alpha\ne F}}^{G}l_\alpha (X_\alpha+Y_\alpha+Z_\alpha)
\end{aligned}
\]
Additional constraints:
\[
\begin{aligned}
\text{s.t.}&
\left\{
\begin{array}{lr}
\sum\limits_{m=1}^{T}U_{Xm}=1\\
\sum\limits_{m=1}^{T}U_{Ym}=1\\
\sum\limits_{m=1}^{T}U_{Zm}=1\\
\sum\limits_{i=1}^3X_{\alpha ij}\le M(1-U_{Xm})\text{ if }l_{\alpha}< f(m,j)  ,\text{ for }\alpha=A,\cdots,G,m=1,\cdots,T,j=1,\cdots,5\\
\sum\limits_{i=1}^3Y_{\alpha ij}\le M(1-U_{Ym})\text{ if }l_{\alpha}< f(m,j)  ,\text{ for }\alpha=A,\cdots,G,m=1,\cdots,T,j=1,\cdots,5\\
\sum\limits_{i=1}^3Z_{\alpha ij}\le M(1-U_{Zm})\text{ if }l_{\alpha}< f(m,j)  ,\text{ for }\alpha=A,\cdots,G,m=1,\cdots,T,j=1,\cdots,5\\
U_{Xm},U_{Ym},U_{Zm} \in \{0,1\}
\end{array}
\right.
\end{aligned}
\]


\[
\begin{aligned}
\min\quad&\sum_{\alpha=A}^{H}\sum_{i=1}^{3}\sum_{j=1}^{5}c_{i}\left(X_{\alpha ij}+Y_{\alpha ij}+Z_{\alpha ij}\right)
+\sum_{\alpha=A}^{H}C_{{\alpha}}\left(X_{\alpha}+Y_{\alpha }+Z_{\alpha }\right)+W(X+Y+Z)\\
&-R\sum_{\substack{\alpha=A\\ \alpha\ne F}}^{G}l_\alpha (X_\alpha+Y_\alpha+Z_\alpha)
\end{aligned}
\]


\section{Introduction}

\lipsum[2]
\begin{itemize}
\item minimizes the discomfort to the hands, or
\item maximizes the outgoing velocity of the ball.
\end{itemize}
We focus exclusively on the second definition.

\begin{itemize}
\item the initial velocity and rotation of the ball,
\item the initial velocity and rotation of the bat,
\item the relative position and orientation of the bat and ball, and
\item the force over time that the hitter hands applies on the handle.
\end{itemize}
\lipsum[3]
\begin{itemize}
\item the angular velocity of the bat,
\item the velocity of the ball, and
\item the position of impact along the bat.
\end{itemize}
\lipsum[4]
\emph{center of percussion} [Brody 1986], \lipsum[5]

\begin{Theorem} \label{thm:latex}
\LaTeX
\end{Theorem}
\begin{Lemma} \label{thm:tex}
\TeX .
\end{Lemma}
\begin{proof}
The proof of theorem.
\end{proof}





\section{Analysis of the Problem}
\begin{figure}[h]
\small
\centering
\includegraphics[width=12cm]{mcmthesis-aaa.eps}
\caption{aa} \label{fig:aa}
\end{figure}

 \eqref{aa}
\begin{equation}
a^2 \label{aa}
\end{equation}

\[
  \begin{pmatrix}{*{20}c}
  {a_{11} } & {a_{12} } & {a_{13} }  \\
  {a_{21} } & {a_{22} } & {a_{23} }  \\
  {a_{31} } & {a_{32} } & {a_{33} }  \\
  \end{pmatrix}
  = \frac{{Opposite}}{{Hypotenuse}}\cos ^{ - 1} \theta \arcsin \theta
\]

00
\[
  p_{j}=\begin{cases} 0,&\text{if $j$ is odd}\\
  r!\,(-1)^{j/2},&\text{if $j$ is even}
  \end{cases}
\]



\[
  \arcsin \theta  =
  \mathop{{\int\!\!\!\!\!\int\!\!\!\!\!\int}\mkern-31.2mu
  \bigodot}\limits_\varphi
  {\mathop {\lim }\limits_{x \to \infty } \frac{{n!}}{{r!\left( {n - r}
  \right)!}}} \eqno (1)
\]

\section{Calculating and Simplifying the Model  }
\lipsum[11]

\section{The Model Results}
\lipsum[6]

\section{Validating the Model}
\lipsum[9]

\section{Conclusions}
\lipsum[6]

\section{A Summary}
\lipsum[6]

\section{Evaluate of the Mode}

\section{Strengths and weaknesses}
\lipsum[12]

\subsection{Strengths}
\begin{itemize}
\item \textbf{Applies widely}\\
This  system can be used for many types of airplanes, and it also
solves the interference during  the procedure of the boarding
airplane,as described above we can get to the  optimization
boarding time.We also know that all the service is automate.
\item \textbf{Improve the quality of the airport service}\\
Balancing the cost of the cost and the benefit, it will bring in
more convenient  for airport and passengers.It also saves many
human resources for the airline. \item \textbf{}
\end{itemize}

\begin{thebibliography}{99}
\bibitem{1} D.~E. KNUTH   The \TeX{}book  the American
Mathematical Society and Addison-Wesley
Publishing Company , 1984-1986.
\bibitem{2}Lamport, Leslie,  \LaTeX{}: `` A Document Preparation System '',
Addison-Wesley Publishing Company, 1986.
\bibitem{3}\url{http://www.latexstudio.net/}
\bibitem{4}\url{http://www.chinatex.org/}
\end{thebibliography}

\begin{appendices}

\section{First appendix}

\lipsum[13]

Here are simulation programmes we used in our model as follow.\\

\textbf{\textcolor[rgb]{0.98,0.00,0.00}{Input matlab source:}}
\lstinputlisting[language=Matlab]{./code/mcmthesis-matlab1.m}

\section{Second appendix}

some more text \textcolor[rgb]{0.98,0.00,0.00}{\textbf{Input C++ source:}}
\lstinputlisting[language=C++]{./code/mcmthesis-sudoku.cpp}

\end{appendices}
\end{document}

%%
%% This work consists of these files mcmthesis.dtx,
%%                                   figures/ and
%%                                   code/,
%% and the derived files             mcmthesis.cls,
%%                                   mcmthesis-demo.tex,
%%                                   README,
%%                                   LICENSE,
%%                                   mcmthesis.pdf and
%%                                   mcmthesis-demo.pdf.
%%
%% End of file `mcmthesis-demo.tex'.
